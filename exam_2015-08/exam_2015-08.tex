\documentclass[]{article}
\usepackage[T1]{fontenc}
\usepackage{lmodern}
\usepackage{amssymb,amsmath}
\usepackage{ifxetex,ifluatex}
\usepackage{fixltx2e} % provides \textsubscript
% use upquote if available, for straight quotes in verbatim environments
\IfFileExists{upquote.sty}{\usepackage{upquote}}{}
\ifnum 0\ifxetex 1\fi\ifluatex 1\fi=0 % if pdftex
  \usepackage[utf8]{inputenc}
\else % if luatex or xelatex
  \ifxetex
    \usepackage{mathspec}
    \usepackage{xltxtra,xunicode}
  \else
    \usepackage{fontspec}
  \fi
  \defaultfontfeatures{Mapping=tex-text,Scale=MatchLowercase}
  \newcommand{\euro}{€}
\fi
% use microtype if available
\IfFileExists{microtype.sty}{\usepackage{microtype}}{}
\usepackage{color}
\usepackage{fancyvrb}
\newcommand{\VerbBar}{|}
\newcommand{\VERB}{\Verb[commandchars=\\\{\}]}
\DefineVerbatimEnvironment{Highlighting}{Verbatim}{commandchars=\\\{\}}
% Add ',fontsize=\small' for more characters per line
\newenvironment{Shaded}{}{}
\newcommand{\KeywordTok}[1]{\textcolor[rgb]{0.00,0.44,0.13}{\textbf{{#1}}}}
\newcommand{\DataTypeTok}[1]{\textcolor[rgb]{0.56,0.13,0.00}{{#1}}}
\newcommand{\DecValTok}[1]{\textcolor[rgb]{0.25,0.63,0.44}{{#1}}}
\newcommand{\BaseNTok}[1]{\textcolor[rgb]{0.25,0.63,0.44}{{#1}}}
\newcommand{\FloatTok}[1]{\textcolor[rgb]{0.25,0.63,0.44}{{#1}}}
\newcommand{\CharTok}[1]{\textcolor[rgb]{0.25,0.44,0.63}{{#1}}}
\newcommand{\StringTok}[1]{\textcolor[rgb]{0.25,0.44,0.63}{{#1}}}
\newcommand{\CommentTok}[1]{\textcolor[rgb]{0.38,0.63,0.69}{\textit{{#1}}}}
\newcommand{\OtherTok}[1]{\textcolor[rgb]{0.00,0.44,0.13}{{#1}}}
\newcommand{\AlertTok}[1]{\textcolor[rgb]{1.00,0.00,0.00}{\textbf{{#1}}}}
\newcommand{\FunctionTok}[1]{\textcolor[rgb]{0.02,0.16,0.49}{{#1}}}
\newcommand{\RegionMarkerTok}[1]{{#1}}
\newcommand{\ErrorTok}[1]{\textcolor[rgb]{1.00,0.00,0.00}{\textbf{{#1}}}}
\newcommand{\NormalTok}[1]{{#1}}
\ifxetex
  \usepackage[setpagesize=false, % page size defined by xetex
              unicode=false, % unicode breaks when used with xetex
              xetex]{hyperref}
\else
  \usepackage[unicode=true]{hyperref}
\fi
\hypersetup{breaklinks=true,
            bookmarks=true,
            pdfauthor={},
            pdftitle={},
            colorlinks=true,
            citecolor=blue,
            urlcolor=blue,
            linkcolor=magenta,
            pdfborder={0 0 0}}
\urlstyle{same}  % don't use monospace font for urls
\setlength{\parindent}{0pt}
\setlength{\parskip}{6pt plus 2pt minus 1pt}
\setlength{\emergencystretch}{3em}  % prevent overfull lines
\setcounter{secnumdepth}{0}

\author{}
\date{}

\begin{document}

\section{Exam for the course DAT171 Object oriented programming in
Python}\label{exam-for-the-course-dat171-object-oriented-programming-in-python}

\textbf{Time:} 18th August 2015 8:30-18:00

\textbf{Teacher:} Mikael Öhman (mobile: 0736 837 674, office: 031 772
1301)

\textbf{Permitted aids:} Cay Horstmann: Python for everyone. Manuals and
lecture notes are available on the computers.

\textbf{Teacher will visit the rooms:} Around 10:00 and 11:30

\textbf{Formalities}: In each file you hand in, you should write your
examination code. Also enter the computer number (from the exam cover).

The documentation and lectures notes are available on
\texttt{C:\textbackslash{}\_\_EXAM\_\_}. Verify that these files are
there immediately. Handing in the code should be done under the same
folder:
\texttt{C:\textbackslash{}\_\_EXAM\_\_\textbackslash{}Assignments\textbackslash{}},
when you are finished. If you do not store the files here, they are not
included in the exam. Save the files for each question in the
appropriate folder, e.g.
\texttt{C:\textbackslash{}\_\_EXAM\_\_\textbackslash{}Assignments\textbackslash{}Question1\textbackslash{}}.
\emph{Make sure you only hand in one solution for each question,
otherwise you will recieve \textbf{zero} points.} Complex sections of
your code should have a descriptive comment of what is achieved.

When you finish the exam you should log out and fill in the (empty) exam
cover page like normal and hand this in at the end of the exam.

To run the Lectures notes in IPython notebook, you must copy the link
from the start menu, and change the ``Start in:'' property in the
shortcut to \texttt{C:\textbackslash{}}.

\textbf{Corrections:} The results will be announced the lastest on 25th
of August on the course homepage. The review will the same day
12:20-13:00 and on the 26th of August 12:20-13:00.

\textbf{Grading:} There is a total of 25 points which yields grades at
the standard 40\%, 60\%, and 80\% limits:

\begin{Shaded}
\begin{Highlighting}[]
\KeywordTok{def} \NormalTok{grade(points):}
    \KeywordTok{if} \NormalTok{points >= }\DecValTok{20}\NormalTok{:}
        \KeywordTok{return} \StringTok{'Grade 5'}
    \KeywordTok{elif} \NormalTok{points >= }\DecValTok{15}\NormalTok{:}
        \KeywordTok{return} \StringTok{'Grade 4'}
    \KeywordTok{elif} \NormalTok{points >= }\DecValTok{10}\NormalTok{:}
        \KeywordTok{return} \StringTok{'Grade 3'}
    \KeywordTok{else}\NormalTok{:}
        \KeywordTok{return} \StringTok{'Fail'}
\end{Highlighting}
\end{Shaded}

\newpage

\section{Question 1 (6p total)}\label{question-1-6p-total}

The ultimate goal is to convert a color image file to a greyscale image
file.

\subsection{Part A (1p)}\label{part-a-1p}

Colors are often defined by red green and blue color channels. One can
compute the luminance (greyscale value) with: \[
luminance = 0.2126\; R + 0.7152\; G + 0.0722\; B
\]

Write a function to converts integer red green and blue color values to
integer greyscale. Example usage

\begin{Shaded}
\begin{Highlighting}[]
\NormalTok{>>> x = luminance(}\DecValTok{40}\NormalTok{, }\DecValTok{200}\NormalTok{, }\DecValTok{130}\NormalTok{) }\CommentTok{# Passing in R = 40, G = 200, B = 130}
\NormalTok{>>> x}
\DecValTok{161}
\end{Highlighting}
\end{Shaded}

\subsection{Part B (5p)}\label{part-b-5p}

The PPM image format (file.ppm) is a simple text based image format

\begin{verbatim}
P3
# The P3 means colors are in ASCII, then 3 columns and 2 rows,
# then 255 for max color, then RGB triplets
3 2
255
103 217 103    43  95 181   219 219 219
244  83 106   255 255 255     0  83 106
\end{verbatim}

The lines starting with \texttt{\#} are commented (ignored). The first
characters \texttt{P3} denote the file format, and these are followd by
the size (width and height). The next number denotes the maximum value
of the channels (in this case, 0 to 255). The rest is the values of the
colors, pixel by pixel (in this case, R=103, G=217, B=103 for the first
pixel)

Linebreaks are optional; any whitespace seperator works. This is also a
valid PPM file:

\begin{verbatim}
P3
3 2
255
103 217 103 43 95 181 219 219 219 244 83 106 255 255
255 0 83 106
\end{verbatim}

The PGM image format looks very similar, but has only 1 channel per
pixel (the luminance / greyscale)

\begin{verbatim}
P2
# Greyscale version
3 2
255
185 90 219
119 255 67
\end{verbatim}

\textbf{Write a function that converts a PPM to PGM file.} Use your
function from the previous question. Your function should handle
arbitrary sizes for the images.

\emph{Hint: Remember to close the files}

\section{Question 2 (4p total)}\label{question-2-4p-total}

You are post-processing some data from an analysis, which requires you
to compute the largest eigenvalue from an analysis. The file
``matrix.csv'' contains the row, column, and values for a matrix. You
must utilize NumPy and SciPy for this.

Write a python script that takes the file ``matrix.csv'' and computes
(and prints) the largest eigenvalue. There is a example file in the exam
folder with a small matrix for you to test your code it, however, you
code should work for any matrix size.

\emph{Hint: \texttt{loadtxt} in NumPy is suitable}. \emph{Hint:
\texttt{scipy.sparse}}. \emph{Hint: \texttt{scipy.sparse.linalg}}.

\section{Question 4 (7p total)}\label{question-4-7p-total}

Write a base class \texttt{BankAccount}, and subclasses
\texttt{SavingsAccount} and \texttt{SpendingAccount}. \emph{You should
implement (or make abstract) methods in the base class, and overload in
the subclasses when necessary.}

Savings account: Accounts are not allowed to be overdrafted (no negative
amount).

Spending account: Accounts are allowed to be overdrafted to -5000 SEK,
with a 10\% overdraft fee (up to 500 SEK).

The following methods should be supported:

\begin{itemize}
\itemsep1pt\parskip0pt\parsep0pt
\item
  \texttt{available\_amount()} returns the value of the account.
\item
  \texttt{withdraw(amount)} withdrawing the given amount from the
  account.
\item
  \texttt{deposit(amount)} deposit the given amount to the account.
\item
  \texttt{apply\_overdraft\_fee()} which withdraws the overdraft when
  called.
\item
  Implement support so that the accounts can be printed; for example:
\end{itemize}

\begin{Shaded}
\begin{Highlighting}[]
\NormalTok{>>> account = SavingsAccount(}\DecValTok{1500}\NormalTok{)}
\NormalTok{>>> }\DataTypeTok{print}\NormalTok{(account)}
\NormalTok{Account: }\DecValTok{1500} \NormalTok{SEK.}
\end{Highlighting}
\end{Shaded}

Write and make appropriate use of a exception
\texttt{OverdraftException} where an order cannot be complied with. Have
the exception contain all relevant information about the failed
transaction.

\section{Question 3 (8p total)}\label{question-3-8p-total}

\emph{Note: I want you to do this exercise without using the preexisting
stuff in SciPy or NumPy to test your object oriented programming skills}

A sparse matrix is a matrix which contains mostly zeros. In those cases,
storing only the positions (row, column) and the corresponding nonzero
value is more efficient.

Write a class \texttt{SparseMatrix} that represents a matrix without
storing the zeros.

\begin{enumerate}
\def\labelenumi{\arabic{enumi}.}
\setcounter{enumi}{-1}
\itemsep1pt\parskip0pt\parsep0pt
\item
  (1p) The initialization should be
  \texttt{SparseMatrix(rows, columns, values, nr, nc)} (see example
  usage).
\item
  (2p) Implement the method \texttt{m.get(row, col)} which returns the
  matrix value at the given position (including zero values).
\item
  (2p) Implement support for creating a transposed copy of the matrix;
  \texttt{m\_t = m.transpose()} \emph{Hint: There is a really simple
  solution}
\item
  (3p) Implement support for scalar product with vector by overloading
  the \texttt{*} operator (both \texttt{\_\_mul\_\_} and
  \texttt{\_\_rmul\_\_}). Code it so that it skips all the zeros of the
  matrix (otherwise, it would be pointless to use a sparse matrix).
\end{enumerate}

Example usage: \[
M = \begin{pmatrix} 3 & 0 & 0 & 0 & 2 \\ 0 & 0 & 4 & 0 & 0 \\ 0 & 0 & 7 & 0 & 0 \\ 0 & 2 & 0 & 0 & 4 \\ 0 & 0 & 0 & 9 & 0\end{pmatrix}
\] \[
M \cdot \begin{pmatrix} 1 \\ 0 \\ 3 \\ 2 \\ 3\end{pmatrix} = \begin{pmatrix} 9 \\ 12 \\ 21 \\ 12 \\ 18 \end{pmatrix}
\] \[
\begin{pmatrix} 1 & 0 & 3 & 2 & 3 \end{pmatrix} \cdot M =  M^T \cdot \begin{pmatrix} 1 \\ 0 \\ 3 \\ 2 \\ 3 \end{pmatrix}
\]

\begin{Shaded}
\begin{Highlighting}[]
\NormalTok{>>> m = SparseMatrix([}\DecValTok{0} \DecValTok{3} \DecValTok{1} \DecValTok{2} \DecValTok{4} \DecValTok{0} \DecValTok{3}\NormalTok{], [}\DecValTok{0} \DecValTok{1} \DecValTok{2} \DecValTok{2} \DecValTok{3} \DecValTok{4} \DecValTok{4}\NormalTok{],}
             \NormalTok{[}\DecValTok{3}\NormalTok{., }\DecValTok{2}\NormalTok{., }\DecValTok{4}\NormalTok{., }\DecValTok{7}\NormalTok{., }\DecValTok{9}\NormalTok{., }\DecValTok{2}\NormalTok{., }\DecValTok{4}\NormalTok{.], }\DecValTok{5}\NormalTok{, }\DecValTok{5}\NormalTok{)}
\NormalTok{>>> m * [}\DecValTok{1}\NormalTok{, }\DecValTok{0}\NormalTok{, }\DecValTok{3}\NormalTok{, }\DecValTok{2}\NormalTok{, }\DecValTok{3}\NormalTok{]}
\NormalTok{[}\DecValTok{9}\NormalTok{, }\DecValTok{12}\NormalTok{, }\DecValTok{21}\NormalTok{, }\DecValTok{12}\NormalTok{, }\DecValTok{18}\NormalTok{]}
\NormalTok{>>> [}\DecValTok{1}\NormalTok{, }\DecValTok{0}\NormalTok{, }\DecValTok{3}\NormalTok{, }\DecValTok{2}\NormalTok{, }\DecValTok{3}\NormalTok{] * m == m.transpose() * [}\DecValTok{1}\NormalTok{, }\DecValTok{0}\NormalTok{, }\DecValTok{3}\NormalTok{, }\DecValTok{2}\NormalTok{, }\DecValTok{3}\NormalTok{]}
\OtherTok{True}
\NormalTok{>>> m.get(}\DecValTok{0}\NormalTok{,}\DecValTok{0}\NormalTok{)}
\FloatTok{3.0}
\NormalTok{>>> m.get(}\DecValTok{1}\NormalTok{,}\DecValTok{5}\NormalTok{)}
\FloatTok{0.0}
\end{Highlighting}
\end{Shaded}

\emph{Hint: You don't need to be concerned with optimal performance,
just that it works}

\end{document}
